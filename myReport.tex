\documentclass[semifinal,isne,survey]{cpecmu}

%% This is a sample document demonstrating how to use the CPECMU
%% project template. If you are having trouble, see "cpecmu.pdf" for
%% documentation.

\projectNo{P069-1}
\acadyear{2025}

\titleTH{การออกแบบพัฒนาระบบความปลอดภัยของข้อมูลและวิเคราะห์เหตุการณ์สำหรับระบบการดูแลสุขภาพ}
\titleEN{Security Information and Event Management for Monitoring Healthcare Systems}

\author{นายเคน ธนภัทร ธาราวัชรศาสตร์}{KEN TANAPAT TARAWATCHARASART}{650615013}
\author{นายยศสวิน เทพคำ}{YOTSAWIN THEPKHAM}{650615029}
\author{นายอัฎฐพร ศรีเขียว}{ATTHAPORN SRIKIAW}{650615039}

\cpeadvisor{sasin}
\cpecommittee{dome}
\cpecommittee{nopparuj}
%\committee{รศ.ดร.\,นิพนธ์ ธีรอำพน}{Assoc.\,Prof.\,Nipon Theera-Umpon, Ph.D.}

%% Some possible packages to include:
\usepackage[final]{graphicx} % for including graphics

%% Add bookmarks and hyperlinks in the document.
\PassOptionsToPackage{hyphens}{url}
\usepackage[colorlinks=true,allcolors=Blue4,citecolor=red,linktoc=all]{hyperref}
\def\UrlLeft#1\UrlRight{$#1$}

%% Needed just by this example, but maybe not by most reports
\usepackage{afterpage} % for outputting
\usepackage{pdflscape} % for landscape figures and tables. 

%% Some other useful packages. Look these up to find out how to use
%% them.
% \usepackage{natbib}    % for author-year citation styles
% \usepackage{txfonts}
% \usepackage{appendix}  % for appendices on a per-chapter basis
% \usepackage{xtab}      % for tables that go over multiple pages
% \usepackage{subfigure} % for subfigures within a figure
% \usepackage{pstricks,pdftricks} % for access to special PostScript and PDF commands
% \usepackage{nomencl}   % if you have a list of abbreviations

%% if you're having problems with overfull boxes, you may need to increase
%% the tolerance to 9999
% \tolerance=9999

\bibliographystyle{plain}
% \bibliographystyle{IEEEbib}

% \renewcommand{\topfraction}{0.85}
% \renewcommand{\textfraction}{0.1}
% \renewcommand{\floatpagefraction}{0.75}

%% Example for glossary entry
%% Need to use glossary option
%% See glossaries package for complete documentation.
\ifglossary
  \newglossaryentry{lorem ipsum}{
    name=lorem ipsum,
    description={derived from Latin dolorem ipsum, translated as ``pain itself''}
  }
\fi

%% Uncomment this command to preview only specified LaTeX file(s)
%% imported with \include command below.
%% Any other file imported via \include but not specified here will not
%% be previewed.
%% Useful if your report is large, as you might not want to build
%% the entire file when editing a certain part of your report.
% \includeonly{chapters/intro,chapters/background}

\begin{document}
\include{chapters/frontmatter}

\pagestyle{empty}\cleardoublepage
\normalspacing \setcounter{page}{1} \pagenumbering{arabic} \pagestyle{cpecmu}

\chapter{\ifenglish Introduction\else บทนำ\fi}

\section{\ifenglish Project rationale\else ที่มาของโครงงาน\fi}
    District-level healthcare facilities face a critical cybersecurity crisis, as they are frequent targets holding sensitive data but lack the financial resources and specialized expertise to defend themselves effectively. This vulnerability is underscored by a history of successful breaches and documented instances where facilities have been unable to properly implement security tools on their own, highlighting a crucial gap in technical expertise rather than just a lack of software. Therefore, this project is essential because it will implement a fully configured, open-source Security Information and Event Management (SIEM) and XDR framework using Wazuh. This approach directly addresses the expertise gap by providing these under-resourced institutions with the cost-effective, enterprise-grade threat detection and incident response capabilities needed to prevent future attacks and safeguard critical patient data across the region.

\section{\ifenglish Objectives\else วัตถุประสงค์ของโครงงาน\fi}
    The primary objective of this project is to design, implement, and validate a customized Wazuh SIEM/XDR solution tailored to the specific security, compliance (HAIT+), and operational needs of a resource-limited district hospital.
\begin{itemize}

    \item {Assert management: To enable comprehensive visibility and monitoring of all IT assets within the hospital's network, including servers, workstations, network devices, and medical equipment, ensuring they are properly secured and compliant with security policies.}
    \item {Implement a Baseline Solution: To deploy and configure the latest stable version of Wazuh (v4.12) in a simulated environment based on the requirements.}
    \item {Create a Deployment Guide: To develop a comprehensive guide for IT personnel in under-resourced hospitals to perform a fresh installation of Wazuh.}
    \item {Enhance System Capability: To create and validate custom decoders and rules that allow Wazuh to ingest and analyze logs from specific, otherwise unsupported, devices commonly found in healthcare settings.}
    \item {Validate and Align: To evaluate the system's effectiveness with simulated attacks and map its capabilities to the HAIT+ compliance framework.}
    
\end{itemize}

\section{\ifenglish Project scope\else ขอบเขตของโครงงาน\fi}
    Develop a Wazuh Platform tailored for district-level hospitals where all related types of logs can be received and monitored within the central dashboard, reinforcing the security of network assets.
    
\subsection{\ifenglish In scope\else ในขอบเขต\fi}
\begin{itemize}
    \item{A Deployed Wazuh Instance: A fully configured, virtualized instance of Wazuh v4.12.}
    \item{Custom Log Integration: The development of custom decoders and rules for one to two predefined, non-standard log sources relevant to a hospital environment.}
    \item{Documentation: The creation of a detailed Deployment Guide document and a Final Project Report that includes the HAIT+ compliance mapping.}
    \item{Attack Simulation: The execution of at least two scripted attack scenarios (e.g., simulated phishing credential entry, brute-force login attempt) to test the system's alerting capabilities.}
\end{itemize}
\subsection{\ifenglish Out of scope\else นอกขอบเขต\fi}
\begin{itemize}
    \item {Automated Response: While Wazuh's active response capabilities may be discussed, the project will not implement automated threat remediation actions. The focus remains on detection and alerting.}
    \item{Live Deployment: The project will not be deployed in any live, operational hospital network. All work will be conducted within a controlled, virtualized test environment.}
\end{itemize}
\section{\ifenglish Expected outcomes\else ประโยชน์ที่ได้รับ\fi}
\begin{itemize}
    \item{Strengthened Data Security: Users will have access to improve data security, reducing the probability of cyber attacks, data breaches, and asset compromise}
    \item{Data-Driven Security Decision Making: With access to historical security data and trends, hospital management can make more informed decisions regarding IT security investments, resource allocation, and policy changes.}
    \item{Proactive Vulnerability Management: The system automatically scans monitored assets for known vulnerabilities (CVEs), providing IT staff with a prioritized list of systems that require patching.}
\end{itemize}

\section{\ifenglish Technology and tools\else เทคโนโลยีและเครื่องมือที่ใช้\fi}
\begin{enumerate}
    \item{PC}
    \item{Github}
    \item{Ubuntu Server VM}
    \item{Wazuh Version 4.12}
    \item{Proxmox Open Source Virtualization Platform}
    \item{VMware eXSI}
    \item{Oracle Server VM}
    \item{Linux Server VM}
\end{enumerate}

\section{\ifenglish Project plan\else แผนการดำเนินงาน\fi}

\begin{plan}{6}{2025}{10}{2025}
    \planitem{6}{2025}{7}{2025}{Project Discussion}
    \planitem{7}{2025}{8}{2025}{Requirement Analysis}
    \planitem{8}{2025}{9}{2025}{System Design}
    \planitem{8}{2025}{9}{2025}{Prototype Implementation}
    \planitem{9}{2025}{10}{2025}{Presentation}
    \planitem{10}{2025}{10}{2025}{Final Report Documentation}
\end{plan}

\section{\ifenglish Roles and responsibilities\else บทบาทและความรับผิดชอบ\fi}
\begin{tabularx}{\textwidth}{ | >{\columncolor{lightgray}}X | X | }
    \hline
    \textbf{Project Discussion} & 
    \begin{tabular}{@{}l@{}} % Using a nested tabular for the list of names
        Ken Tanapat Tarawatcharasart \\
        Atthaporn Srikiaw \\
        Yotsawin Thepkham
    \end{tabular} \\
    \hline
    \textbf{Requirement Analysis} & 
    \begin{tabular}{@{}l@{}}
        Atthaporn Srikiaw \\
        Yotsawin Thepkham
    \end{tabular} \\
    \hline
    \textbf{System Design} & 
    \begin{tabular}{@{}l@{}} % Using a nested tabular for the list of names
        Ken Tanapat Tarawatcharasart \\
        Atthaporn Srikiaw \\
    \end{tabular} \\
    \hline
    \textbf{Prototype Implementation} & 
    \begin{tabular}{@{}l@{}} % Using a nested tabular for the list of names
        Ken Tanapat Tarawatcharasart \\
        Atthaporn Srikiaw \\
    \end{tabular} \\
    \hline
    \textbf{Presentation} & 
    \begin{tabular}{@{}l@{}} % Using a nested tabular for the list of names
        Ken Tanapat Tarawatcharasart \\
        Atthaporn Srikiaw \\
        Yotsawin Thepkham
    \end{tabular} \\
    \hline
    \textbf{Final Report Documentation} & 
    \begin{tabular}{@{}l@{}} % Using a nested tabular for the list of names
        Ken Tanapat Tarawatcharasart \\
        Atthaporn Srikiaw \\
        Yotsawin Thepkham
    \end{tabular} \\
    \hline
\end{tabularx}
\section{\ifenglish%
Impacts of this project on society, health, safety, legal, and cultural issues
\else%
ผลกระทบด้านสังคม สุขภาพ ความปลอดภัย กฎหมาย และวัฒนธรรม
\fi}

This project aims to directly address the health and safety of patients by securing the digital infrastructure of healthcare facilities by strengthening the data security of each hospital's IT assets. This helps to prevent compromise of assets, which can lead to the malfunction of critical medical devices and directly endanger patients. Additionally, it provides cost-effective and enterprise-grade threat detection capabilities that address the financial and expertise gap, where it often leaves the smaller public healthcare facilities vulnerable.

\section{\ifenglish Budget Plan\else แผนการใช้งบประมาณ\fi}
This project aims to be cost-effective by utilizing university resources and open-source software whenever feasible. Nonetheless, a budget is necessary to address essential project-related expenses. These expenses encompass necessary travel to district hospitals for on-site consultations and data collection, subscriptions for collaborative documentation software, and materials for the final project presentation. A detailed breakdown of the anticipated costs is provided in the table.

\begin{table}[h!]
\centering
\caption{Estimated Project Budget}
\label{tab:budget}
\begin{tabularx}{\linewidth}{@{} l >{\raggedright\arraybackslash}X c r r @{}}
\toprule
\textbf{Category} & \textbf{Item Description} & \textbf{Qty.} & \textbf{Unit Cost (THB)} & \textbf{Total (THB)} \\ \midrule
\textbf{Travel} & Round trip to Doi Saket Hospital & 1 & 1,100.00 & 1,100.00 \\
 & Round trip to Mae Wang Hospital & 1 & 1,150.00 & 1,150.00 \\ \addlinespace
\textbf{Software} & Overleaf Student Plan Subscription for collaborative authoring & 2 months & 260.00 & 520.00 \\ \addlinespace
\textbf{Presentation} & A1-sized color poster printing for final project showcase & 1 & 350.00 & 350.00 \\ \midrule[\heavyrulewidth]
\multicolumn{4}{r}{\textbf{Estimated Grand Total}} & \textbf{3,120.00} \\ \bottomrule
\end{tabularx}
\end{table}
\chapter{\ifenglish Background Knowledge and Theory\else ทฤษฎีที่เกี่ยวข้อง\fi}

This chapter provides a comprehensive review of the foundational concepts related to Security Information and Event Management (SIEM). It begins by defining SIEM and its importance in modern cybersecurity, then delves into the core components and operational workflow of a typical SIEM system. Furthermore, it examines the specific applications and challenges of implementing SIEM solutions within the healthcare sector, concluding with a brief overview of prominent technologies in the field.

% --- Section 2.1: Defining SIEM ---
\section{Security Information and Event Management}
\label{sec:what-is-siem}
Security Information and Event Management (SIEM) represents a comprehensive approach to security by combining the functions of Security Information Management (SIM) and Security Event Management (SEM). The primary goal of a SIEM system is to provide a holistic view of an organization's information security posture by collecting, aggregating, and analyzing log data from a multitude of sources in real-time. This centralized analysis enables security teams to detect potential threats, investigate security incidents, and generate reports for compliance purposes. According to the National Institute of Standards and Technology (NIST), effective log management, which is the core of SIEM, is essential for identifying and responding to security incidents in a timely manner \cite{nist_sp800-92}.


% --- Section 2.2: Core Components of a SIEM System ---
\section{Core Components of a SIEM System}
\subsection{Log Collection and Aggregation}
\label{sec:siem-components}
The foundational stage of any SIEM operation is the collection of log and event data from a wide array of sources across the IT infrastructure. These sources, often referred to as log sources, include network devices (firewalls, routers, switches), servers (Windows, Linux), applications (databases, web servers), and security solutions (antivirus, intrusion detection systems). The SIEM system uses agents or agentless collection methods to pull this data into a central repository. This aggregation is critical, as it creates a single, unified dataset for analysis.\cite{chapple_cybersecurity}

\subsection{Data Normalization and Parsing}
Log data from different vendors and systems comes in a multitude of proprietary and standard formats. For the SIEM to analyze this disparate information, it must first be normalized. Normalization is the process of parsing the raw log data and converting it into a common, structured format. For example, usernames, IP addresses, and event timestamps are extracted from various log formats and mapped to standard fields. This ensures that an event from a firewall can be directly compared and correlated with an event from a Windows server.

\subsubsection{Threat Detection and Log Correlation}
This is the core analytical engine of the SIEM. The system employs a correlation engine that uses a set of rules to analyze the normalized data in real-time. These rules are designed to identify patterns indicative of suspicious or malicious activity that would be invisible when looking at logs from a single source. For instance, a correlation rule might trigger an alert if it detects a user logging in from two different geographical locations within an impossible timeframe, or if it sees a brute-force attack on a server followed by a successful login and data exfiltration to an external IP address.


\subsubsection{Alerting and Incident Response}
When the correlation engine identifies a potential security incident, it generates an alert. These alerts are presented to security analysts through a dashboard or console and are typically prioritized based on severity (e.g., Low, Medium, High, Critical). This allows analysts to focus on the most pressing threats first. Modern SIEMs often integrate with Security Orchestration, Automation, and Response (SOAR) platforms to automate initial response actions, such as blocking an IP address at the firewall or disabling a compromised user account.

\subsection{Reporting and Compliance}
SIEM systems are instrumental in meeting regulatory compliance requirements, such as HIPAA, GDPR, and PCI-DSS. They provide long-term storage of log data for forensic analysis and can automatically generate reports tailored to specific compliance frameworks. These reports can be used to demonstrate to auditors that the organization has the necessary security controls in place for monitoring, alerting, and responding to security events.


% --- Section 2.3: SIEM in the Healthcare Sector ---
\section{SIEM in the Healthcare Sector}
\label{sec:siem-in-healthcare}
The healthcare sector presents a unique and highly challenging environment for cybersecurity. Healthcare organizations are prime targets for cyberattacks due to the high value of the data they possess, known as Protected Health Information (PHI). This data, which includes medical records, personal identification, and financial information, is lucrative on the black market. Furthermore, the IT infrastructure within hospitals is often a complex mix of standard corporate systems, specialized medical devices (such as MRI machines and infusion pumps), and Internet of Things (IoT) devices, many of which may have outdated operating systems and limited security capabilities \cite{kruse_healthcare_cybersecurity}.

Implementing a SIEM solution is critical for healthcare organizations to gain visibility into this complex environment. A SIEM can help by:
\begin{itemize}
    \item \textbf{Protecting Patient Data:} By monitoring access to databases and systems containing PHI, a SIEM can detect and alert on unauthorized access attempts or unusual data exfiltration patterns.
    \item \textbf{Securing Medical Devices:} It can monitor network traffic from medical devices that cannot have security agents installed, identifying anomalous behavior that might indicate a compromise.
    \item \textbf{Ensuring HAIT+ Compliance:} The Hospital IT Accreditation+ mandates strict security controls for protecting patient data. A SIEM provides the necessary audit trails, log retention, and real-time monitoring capabilities required to demonstrate compliance to auditors. For instance, it can track every instance a patient's record is accessed, providing an essential tool for privacy and security officers.
\end{itemize}

% --- Section 2.4: Review of Existing SIEM Solutions ---
\section{Review of Existing SIEM Solutions}
The SIEM market consists of a wide range of solutions, from large-scale, enterprise-grade commercial platforms to flexible and highly customizable open-source projects. Understanding the landscape of these tools is essential for making informed decisions about security architecture.
\subsection{Commercial SIEM Platforms}
Commercial solutions are known for their comprehensive feature sets, dedicated support, and extensive integration capabilities. The leaders in this space are consistently evaluated by industry analysts based on their ability to execute and completeness of vision \cite{gartner_siem_2022}. Two of the most prominent are:
\begin{itemize}
    \item \textbf{Splunk Enterprise Security:} Often regarded as a market leader, Splunk is a powerful data platform with a specialized security module. It is renowned for its highly flexible Search Processing Language (SPL), which allows for deep data investigation, and its powerful visualization and dashboarding capabilities.
    \item \textbf{IBM QRadar:} A mature and feature-rich SIEM platform that focuses heavily on security intelligence. It integrates real-time threat intelligence feeds and provides strong analytics for correlating events to identify complex, multi-stage attacks.
\end{itemize}

\subsection{Open-Source SIEM Solutions}
Open-source tools offer a cost-effective and highly customizable alternative to commercial products, making them popular for organizations with strong in-house technical expertise.
\begin{itemize}
    \item \textbf{The ELK Stack:} This is not a SIEM out-of-the-box but a powerful combination of three open-source projects that form the foundation of many modern SIEMs: \textbf{E}lasticsearch (a search and analytics engine), \textbf{L}ogstash (a data processing pipeline), and \textbf{K}ibana (a data visualization tool). It requires significant customization to function as a full SIEM. \cite{elastic_stack}
    \item \textbf{Wazuh:} A popular open-source security platform that provides SIEM and Extended Detection and Response (XDR) capabilities. It is built upon the ELK Stack and offers a comprehensive solution for threat detection, integrity monitoring, incident response, and compliance, making it a powerful free alternative to commercial platforms \cite{wazuh_docs}.
\end{itemize}



% --- Section 2.5: ISNE knowledge used, applied, or integrated in this project ---

\section{\ifenglish%
\ifcpe CPE \else ISNE \fi knowledge used, applied, or integrated in this project
\else%
ความรู้ตามหลักสูตรซึ่งถูกนำมาใช้หรือบูรณาการในโครงงาน
\fi
}

\label{sec:coursework-integration}

This project integrates key concepts and practical skills acquired from several core courses in the Information Systems and Network Engineering curriculum. The following areas of study were instrumental in the planning and design of this SIEM solution.

\begin{description}
    \item[Project Management] Applied in planning the system development process by setting clear objectives, organizing tasks, and monitoring progress to keep the project on schedule. This knowledge ensured a structured workflow, adherence to timelines, and the timely delivery of project milestones.

    \item[Network Design] A strong understanding of network design was crucial for defining the assets within the hospital's network topology. This knowledge provides the foundation for identifying and managing assets effectively and ensures that the SIEM system is designed to be fully compatible with the existing network infrastructure.

    \item[Asset Management] Learned in the Cybersecurity course, asset management is a fundamental process for this project. After obtaining a list of critical assets, we will evaluate and prioritize them to ensure the SIEM's monitoring capabilities are focused on the most valuable and sensitive resources within the healthcare environment.
    
\end{description}
% --- Section 2.6: Extracurricular knowledge used, applied, or integrated in this project ---
\break
\section{\ifenglish%
Extracurricular knowledge used, applied, or integrated in this project
\else%
ความรู้นอกหลักสูตรซึ่งถูกนำมาใช้หรือบูรณาการในโครงงาน
\fi
}

\label{sec:extracurricular-knowledge}

The successful implementation of this project also relied heavily on knowledge and skills acquired through self-study and practical experience with technologies outside the formal curriculum.


\begin{description}
    \item[Wazuh] Implemented as the core SIEM platform, Wazuh was used for log collection, analysis, and real-time alerting. It actively monitored server and network activity, detected anomalies, and generated alerts for suspicious events. By applying Wazuh, the project ensured that critical data within the test environment was continuously monitored, improving security visibility and enabling a rapid response capability to potential incidents.

    \item[Proxmox] This virtualization management platform was used to create and manage the virtual machines for the entire project. Proxmox provided a safe, isolated environment for testing and deploying the Wazuh system without impacting any live systems. The platform's support for snapshots, backups, and resource management was instrumental in ensuring the reliability and efficiency of the development and testing environment.

    \item[OPNsense] As an open-source firewall and network monitoring platform, OPNsense was used during the system evaluation phase. It enabled the simulation of realistic network traffic, controlled network access, and generated syslog data. This enabled thorough testing to verify that the Wazuh system correctly ingested firewall logs and triggered appropriate alerts, validating the system’s responsiveness in a controlled setting.
\end{description}
\chapter{\ifproject%
\ifenglish Project Structure and Methodology\else โครงสร้างและขั้นตอนการทำงาน\fi
\else%
\ifenglish Project Structure\else โครงสร้างของโครงงาน\fi
\fi
}

ในบทนี้จะกล่าวถึงหลักการ และการออกแบบระบบ

\makeatletter

% \renewcommand\section{\@startsection {section}{1}{\z@}%
%                                    {13.5ex \@plus -1ex \@minus -.2ex}%
%                                    {2.3ex \@plus.2ex}%
%                                    {\normalfont\large\bfseries}}

\makeatother
%\vspace{2ex}
% \titleformat{\section}{\normalfont\bfseries}{\thesection}{1em}{}
% \titlespacing*{\section}{0pt}{10ex}{0pt}

\section{Alice in Wonderland}

\begin{figure}
\begin{center}
\includegraphics{800px-Briny_Beach.jpg}
\end{center}
\caption[Poem]{The Walrus and the Carpenter}
\label{fig:walrus}
\end{figure}

\subsection{The Black Kitten}
  One thing was certain, that the WHITE kitten had had nothing to
do with it:---it was the black kitten's fault entirely~\cite{aiw}.  For the
white kitten had been having its face washed by the old cat for
the last quarter of an hour (and bearing it pretty well,
considering); so you see that it COULDN'T have had any hand in
the mischief.

  The way Dinah washed her children's faces was this:  first she
held the poor thing down by its ear with one paw, and then with
the other paw she rubbed its face all over, the wrong way,
beginning at the nose:  and just now, as I said, she was hard at
work on the white kitten, which was lying quite still and trying
to purr---no doubt feeling that it was all meant for its good.

  But the black kitten had been finished with earlier in the
afternoon, and so, while Alice was sitting curled up in a corner
of the great arm-chair, half talking to herself and half asleep,
the kitten had been having a grand game of romps with the ball of
worsted Alice had been trying to wind up, and had been rolling it
up and down till it had all come undone again; and there it was,
spread over the hearth-rug, all knots and tangles, with the
kitten running after its own tail in the middle.

\subsection{The Reproach}

  `Oh, you wicked little thing!' cried Alice, catching up the
kitten, and giving it a little kiss to make it understand that it
was in disgrace.  `Really, Dinah ought to have taught you better
manners!  You OUGHT, Dinah, you know you ought!' she added,
looking reproachfully at the old cat, and speaking in as cross a
voice as she could manage---and then she scrambled back into the
arm-chair, taking the kitten and the worsted with her, and began
winding up the ball again.  But she didn't get on very fast, as
she was talking all the time, sometimes to the kitten, and
sometimes to herself.  Kitty sat very demurely on her knee,
pretending to watch the progress of the winding, and now and then
putting out one paw and gently touching the ball, as if it would
be glad to help, if it might.

  `Do you know what to-morrow is, Kitty?' Alice began.  `You'd
have guessed if you'd been up in the window with me---only Dinah
was making you tidy, so you couldn't.  I was watching the boys
getting in stick for the bonfire---and it wants plenty of
sticks, Kitty!  Only it got so cold, and it snowed so, they had
to leave off.  Never mind, Kitty, we'll go and see the bonfire
to-morrow.'  Here Alice wound two or three turns of the worsted
round the kitten's neck, just to see how it would look:  this led
to a scramble, in which the ball rolled down upon the floor, and
yards and yards of it got unwound again.

  `Do you know, I was so angry, Kitty,' Alice went on as soon as
they were comfortably settled again, `when I saw all the mischief
you had been doing, I was very nearly opening the window, and
putting you out into the snow!  And you'd have deserved it, you
little mischievous darling!  What have you got to say for
yourself?  Now don't interrupt me!' she went on, holding up one
finger.  `I'm going to tell you all your faults.  Number one:
you squeaked twice while Dinah was washing your face this
morning.  Now you can't deny it, Kitty:  I heard you!  What that
you say?' (pretending that the kitten was speaking.)  `Her paw
went into your eye?  Well, that's YOUR fault, for keeping your
eyes open---if you'd shut them tight up, it wouldn't have
happened.  Now don't make any more excuses, but listen!  Number
two:  you pulled Snowdrop away by the tail just as I had put down
the saucer of milk before her!  What, you were thirsty, were you?

\chapter{\ifproject%
\ifenglish Experimentation and Results\else การทดลองและผลลัพธ์\fi
\else%
\ifenglish System Evaluation\else การประเมินระบบ\fi
\fi}
\label{ch:evaluation}

% --- Section 4.1: Test Environment ---
\section{Test Environment Setup}
\label{sec:test-env}
The evaluation was conducted in the virtual lab environment built on the Proxmox VE platform. The environment consisted of a central Wazuh server running on Ubuntu Server, an endpoint virtual machine running Windows 11, and a second endpoint virtual machine running Ubuntu Server. The OPNsense firewall was used as the representative network device for generating syslog data for the firewall-related test case.


% --- Section 4.2: Test Case Execution and Results ---
\section{Test Case Execution and Results}
\label{sec:test-results}
The four test cases defined in the methodology were executed sequentially. The success of each test was determined by the generation of a timely, accurate, and actionable alert within the Wazuh Dashboard.

\subsection{Test Case 1: Brute-Force Attack Detection}
\begin{itemize}
    \item \textbf{Objective:} To verify the system's ability to detect and alert on a brute-force SSH login attempt against the Ubuntu server endpoint.
    \item \textbf{Methodology:} A script was executed to attempt multiple SSH logins to the Ubuntu server with incorrect passwords over a short period.
    \item \textbf{Results:} ...
    Figure \ref{fig:brute-force-alert}
\end{itemize}

\subsection{Test Case 2: File Integrity Monitoring}
\begin{itemize}
    \item \textbf{Objective:} To validate the File Integrity Monitoring (FIM) module's ability to detect unauthorized modifications to critical system files on the Windows 11 endpoint.
    \item \textbf{Methodology:} The C:Windows-System32-drivers-etc-hosts file on the Windows 11 machine was manually edited and saved.
    \item \textbf{Results:} ...
\end{itemize}

\subsection{Test Case 3: Firewall Log Analysis}
\begin{itemize}
    \item \textbf{Objective:} To ensure the SIEM could successfully ingest, parse, and alert on syslog data from a network firewall.
    \item \textbf{Methodology:} A rule was configured on the OPNsense firewall to block all traffic from a specific test IP address. Traffic was then generated from that IP address, triggering the "deny" rule.
    \item \textbf{Results:} ...
\end{itemize}
\subsection{Test Case 4: Custom Log Analysis}
\begin{itemize}
    \item \textbf{Objective:} To verify the functionality of the custom-developed decoder for a non-standard log format.
    \item \textbf{Methodology:} A log entry matching the custom format was manually written to a monitored log file on the Ubuntu server. The log simulated a hypothetical application event, such as "PatientRecordAccessed: User 'testuser' accessed record '12345'".
    \item \textbf{Results:} \ref{fig:custom-log-alert}. 
\end{itemize}
\ifproject
\include{chapters/conclusion}
\fi

\bibliography{sampleReport}

\ifproject
\normalspacing
\appendix
\include{chapters/appendix}

%% Display glossary (optional) -- need glossary option.
\ifglossary\glossarypage\fi

%% Display index (optional) -- need idx option.
\ifindex\indexpage\fi

\begin{biosketch}
\begin{center}
  \includegraphics[width=1.5in]{mugshot.jpg}
\end{center}
Your biosketch goes here. Make sure it sits inside
the \texttt{biosketch} environment.
\end{biosketch}
\fi % \ifproject
\end{document}
