\chapter{\ifenglish Introduction\else บทนำ\fi}

\section{\ifenglish Project rationale\else ที่มาของโครงงาน\fi}
    District-level healthcare facilities face a critical cybersecurity crisis, as they are frequent targets holding sensitive data but lack the financial resources and specialized expertise to defend themselves effectively. This vulnerability is underscored by a history of successful breaches and documented instances where facilities have been unable to properly implement security tools on their own, highlighting a crucial gap in technical expertise rather than just a lack of software. Therefore, this project is essential because it will implement a fully configured, open-source Security Information and Event Management (SIEM) and XDR framework using Wazuh. This approach directly addresses the expertise gap by providing these under-resourced institutions with the cost-effective, enterprise-grade threat detection and incident response capabilities needed to prevent future attacks and safeguard critical patient data across the region.

\section{\ifenglish Objectives\else วัตถุประสงค์ของโครงงาน\fi}
    The primary objective of this project is to design, implement, and validate a customized Wazuh SIEM/XDR solution tailored to the specific security, compliance (HAIT+), and operational needs of a resource-limited district hospital.
\begin{itemize}

    \item {Assert management: To enable comprehensive visibility and monitoring of all IT assets within the hospital's network, including servers, workstations, network devices, and medical equipment, ensuring they are properly secured and compliant with security policies.}
    \item {Implement a Baseline Solution: To deploy and configure the latest stable version of Wazuh (v4.12) in a simulated environment based on the requirements.}
    \item {Create a Deployment Guide: To develop a comprehensive guide for IT personnel in under-resourced hospitals to perform a fresh installation of Wazuh.}
    \item {Enhance System Capability: To create and validate custom decoders and rules that allow Wazuh to ingest and analyze logs from specific, otherwise unsupported, devices commonly found in healthcare settings.}
    \item {Validate and Align: To evaluate the system's effectiveness with simulated attacks and map its capabilities to the HAIT+ compliance framework.}
    
\end{itemize}

\section{\ifenglish Project scope\else ขอบเขตของโครงงาน\fi}
    Develop a Wazuh Platform tailored for district-level hospitals where all related types of logs can be received and monitored within the central dashboard, reinforcing the security of network assets.
    
\subsection{\ifenglish In scope\else ในขอบเขต\fi}
\begin{itemize}
    \item{A Deployed Wazuh Instance: A fully configured, virtualized instance of Wazuh v4.12.}
    \item{Custom Log Integration: The development of custom decoders and rules for one to two predefined, non-standard log sources relevant to a hospital environment.}
    \item{Documentation: The creation of a detailed Deployment Guide document and a Final Project Report that includes the HAIT+ compliance mapping.}
    \item{Attack Simulation: The execution of at least two scripted attack scenarios (e.g., simulated phishing credential entry, brute-force login attempt) to test the system's alerting capabilities.}
\end{itemize}
\subsection{\ifenglish Out of scope\else นอกขอบเขต\fi}
\begin{itemize}
    \item {Automated Response: While Wazuh's active response capabilities may be discussed, the project will not implement automated threat remediation actions. The focus remains on detection and alerting.}
    \item{Live Deployment: The project will not be deployed in any live, operational hospital network. All work will be conducted within a controlled, virtualized test environment.}
\end{itemize}
\section{\ifenglish Expected outcomes\else ประโยชน์ที่ได้รับ\fi}
\begin{itemize}
    \item{Strengthened Data Security: Users will have access to improve data security, reducing the probability of cyber attacks, data breaches, and asset compromisation}
    \item{Data-Driven Security Decision Making: With access to historical security data and trends, hospital management can make more informed decisions regarding IT security investments, resource allocation, and policy changes.}
    \item{Proactive Vulnerability Management: The system automatically scans monitored assets for known vulnerabilities (CVEs), providing IT staff with a prioritized list of systems that require patching.}
\end{itemize}

\section{\ifenglish Technology and tools\else เทคโนโลยีและเครื่องมือที่ใช้\fi}
\begin{enumerate}
    \item{PC}
    \item{Github}
    \item{Ubuntu Server VM}
    \item{Wazuh Version 4.12}
    \item{Proxmox Open Source Virtualization Platform}
    \item{VMware eXSI}
    \item{Oracle Server VM}
    \item{Linux Server VM}
\end{enumerate}

\section{\ifenglish Project plan\else แผนการดำเนินงาน\fi}

\begin{plan}{6}{2025}{10}{2025}
    \planitem{6}{2025}{7}{2025}{Project Discussion}
    \planitem{7}{2025}{8}{2025}{Requirement Analysis}
    \planitem{8}{2025}{9}{2025}{System Design}
    \planitem{8}{2025}{9}{2025}{Prototype Implementation}
    \planitem{9}{2025}{10}{2025}{Presentation}
    \planitem{10}{2025}{10}{2025}{Final Report Documentation}
\end{plan}

\section{\ifenglish Roles and responsibilities\else บทบาทและความรับผิดชอบ\fi}
\begin{tabularx}{\textwidth}{ | >{\columncolor{lightgray}}X | X | }
    \hline
    \textbf{Project Discussion} & 
    \begin{tabular}{@{}l@{}} % Using a nested tabular for the list of names
        Ken Tanapat Tarawatcharasart \\
        Atthaporn Srikiaw \\
        Yotsawin Thepkham
    \end{tabular} \\
    \hline
    \textbf{Requirement Analysis} & 
    \begin{tabular}{@{}l@{}}
        Atthaporn Srikiaw \\
        Yotsawin Thepkham
    \end{tabular} \\
    \hline
    \textbf{System Design} & 
    \begin{tabular}{@{}l@{}} % Using a nested tabular for the list of names
        Ken Tanapat Tarawatcharasart \\
        Atthaporn Srikiaw \\
    \end{tabular} \\
    \hline
    \textbf{Prototype Implementation} & 
    \begin{tabular}{@{}l@{}} % Using a nested tabular for the list of names
        Ken Tanapat Tarawatcharasart \\
        Atthaporn Srikiaw \\
    \end{tabular} \\
    \hline
    \textbf{Presentation} & 
    \begin{tabular}{@{}l@{}} % Using a nested tabular for the list of names
        Ken Tanapat Tarawatcharasart \\
        Atthaporn Srikiaw \\
        Yotsawin Thepkham
    \end{tabular} \\
    \hline
    \textbf{Final Report Documentation} & 
    \begin{tabular}{@{}l@{}} % Using a nested tabular for the list of names
        Ken Tanapat Tarawatcharasart \\
        Atthaporn Srikiaw \\
        Yotsawin Thepkham
    \end{tabular} \\
    \hline
\end{tabularx}
\section{\ifenglish%
Impacts of this project on society, health, safety, legal, and cultural issues
\else%
ผลกระทบด้านสังคม สุขภาพ ความปลอดภัย กฎหมาย และวัฒนธรรม
\fi}

This project's aims to directly to the health and safety of patients by securing the digital infrastructure of healthcare facilities by strengthening the data security of each hospital's IT assets. This helps to prevent compromisation of assets which can lead to the malfunction of critical medical devices and directly endanger patients. Additionally, it provides cost-effective and enterprise-grade threat detection capabilities that addresses the financial and expertise gap, where it often leaves the smaller public healthcare facilities vulnerable.
